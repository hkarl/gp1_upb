* Aufgabe 1

Die Variable a und b sind boolsche Variablen unbekannter Belegung. 

1) Verkürzen Sie folgende boolsche Ausdrücke, soweit wie möglich: 

 (not(a and a)) and False
 ( (True | b ) | ( False | b) )
 
 2) Nehmen Sie an, dass die Variablen a, b und c die folgende Werten haben: 
 
 a = True
 b = True
 c = False
 
Berechnen Sie die folgende Ausdrücke:
i) b and c
ii) b or c
iii) not a and c
iv) (b and c) or not a

* Aufgabe 2

Mit Python kann man mathematische Ausdrücken auswerten lassen. Sie unterstützt die folgende mathematische Operatoren: +, -, *, /, Potenz (**) und Modulus (\%).   
1) Werten Sie die folgende Ausdrücke in Python aus: 
a) 7/2, 7/2.0 und 7.0/2
b) 3*(1/2) und 3*(1/2.0)
c) 3**2, 3.0**2 und 3**2.0
2) Warum bekommt man unterschiedliche Ergebnissen, obwohl alle Einträgen mathematisch identisch sind? Woran liegt das?
3) Berechnen Sie die folgende Ausdrücke in Python: 
\begin{align}
&\frac{3 \cdot 6}{3 + 1}\\
&\sqrt{6+10} \cdot 3 \\
&5 \text{mod} 4 \\
&(3-5)^4
\end{align}

* Aufgabe 3

1) Welche der folgenden Namen sind legal in Python? Begründen Sie Ihre Antwort: 

or
_or
var
var2
2var
my_var
dein-var

2) Nennen Sie die folgende Variabeltypen: 

a = 4
b = 4.0
c = '4.0'
d = False
e = 'True'
f = 'Test'

Hinweis: Sie können die Fragen auch durch Python verifizieren indem Sie type(x) benutzen. Versuchen Sie aber zuerst ohne Hilfe von Python die Fragen zu beantworten.   

* Aufgabe 4

Sie möchten sich einen Fahrschein am Automaten kaufen, welcher jede Art von Euro-Münzen annimt. Schreiben Sie einen
Programmablaufplan und den entsprechenden Pseudocode, welches zu Ihrem eingegebenen Betrag die Mindestanzahl der Münzen nennt. 
Hinweis: Nutzen Sie das Modulus Operation sowie die folgende Links: https://de.wikipedia.org/wiki/Programmablaufplan und https://de.wikipedia.org/wiki/Pseudocode.

* Aufgabe 5

Sie haben um 18 Uhr ein Termin mit Ihren Kollege vereinbart. Ist er/sie nicht bis 17.55 Uhr bei Ihnen, sagen Sie das Termin ab. Ansosten nehmen Sie teil. Drücken sie die o.g. Handlung in einem Programmablaufplan und Pseudocode aus.